\documentclass[14pt,DIV14]{scrartcl}
\usepackage[cp1251]{inputenc}
\usepackage[russian]{babel}
%\usepackage[utf8]{inputenc}
\usepackage{sheet}
\usepackage{multicol}
 \usepackage{epigraph}
  \usepackage{graphicx}
%\usepackage{xfrac}

\pagestyle{empty}
%\topmargin=-2cm \textheight=25cm
%\evensidemargin=-0,3cm \oddsidemargin=-0,3cm
%\textwidth=16cm

\begin{document}
\sheet{Малый механико-математический факультет, 12-26б}{
Странный разнобой}{дата  2012 7 класс}

\begin{task}
Десять футбольных команд сыграли каждая с каждой по одному разу. В результате у каждой команды оказалось ровно по $x$  очков. Каково наибольшее возможное значение $x$? (Победа~--- 3 очка, ничья~--- 1 очко, поражение~--- 0.)
\end{task}

\begin{task}
На доске написано число 120. За одну операцию число на доске можно либо умножить на простое число, либо разделить на квадрат натурального числа (если делится нацело). За какое наименьшее количество операций можно получить на доске число 1?
\end{task}

\begin{task}
В заповеднике растут деревья, возраст каждого из которых измеряется натуральным числом. Средний возраст деревьев был равен 41 году. После того, как молния сожгла дерево, которому было 2010 лет, средний возраст уцелевших деревьев составил 40 лет. Какое наибольшее количество 2010-летних деревьев еще могло остаться в заповеднике?
\end{task}

\begin{task}
В ряд стоят числа от 1 до 2012 в каком-то порядке. Можно менять местами два соседних числа тогда и только тогда, когда разность большего и меньшего из них не превосходит $n$. При каком наименьшем $n$ эти числа несколькими такими обменами заведомо можно переставить в обратном порядке (независимо от того, как они располагались сначала)?
\end{task}

\begin{task}
Перед Гарри Поттером в ряд лежат несколько шариков, на которых написаны ненулевые цифры (на каждом шарике одна цифра). За один взмах волшебной палочки он может удалить самый левый шарик, при этом после каждого шарика с цифрой $k$ появятся шарики с цифрами $k+1, k+2, ..., 9$. Например, если перед Гарри лежат шарики 2, 6, 5, 9, после взмаха палочки будут шарики 6, 7, 8, 9, 5, 6, 7, 8, 9, 9. Всегда ли Гарри сможет убрать все шарики?
\end{task}

\end{document}
\begin{definition}
\end{definition}
\begin{task}
\end{task}
\begin{axiom}
\end{axiom}
