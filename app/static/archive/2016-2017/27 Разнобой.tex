\documentclass[14pt,DIV14]{scrartcl}
\usepackage[cp1251]{inputenc}
\usepackage[russian]{babel}
%\usepackage[utf8]{inputenc}
\usepackage{sheet}
\usepackage{multicol}
 \usepackage{epigraph}
  \usepackage{graphicx}
%\usepackage{xfrac}
\usepackage{a4wide}
\pagestyle{empty}
\usepackage{tikz}
\usepackage{graphicx,epstopdf}
\def\mod_#1 {\mathrel{\mathop{\equiv}\limits_{#1}}}
\newtheorem{theorem}{Теорема}
\evensidemargin=-1cm \oddsidemargin=-1cm \textwidth=18cm
\topmargin=-2cm \textheight=26cm

\begin{document}
\sheet{Малый механико-математический факультет, 13-02.}{Разнобой
}{1 апреля 2017 9-11 класс}

\begin{task}
Напомним, что число $e=\lim\limits_{n->\infty}\sum\limits_{i=0}^n \dfrac{1}{i!}$. Докажите, что $e$ иррационально.
\end{task}

\begin{task}
Все коэффициенты непостоянного многочлена~--- целые числа по модулю, не превосходящие 2015. Докажите, что любой положительный корень многочлена больше чем $\dfrac{1}{2016}$  
\end{task}



\begin{task}
Пусть $n$~--- натуральное число. На $2n + 1$ карточках написано по ненулевому целому числу; сумма всех чисел также ненулевая. Требуется этими карточками заменить звёздочки в выражении $*x^{2n} + *x^{2n-1} + \ldots + *x + *$ так, чтобы полученный многочлен не имел целых корней. Обязательно ли это можно сделать?
\end{task}



\begin{task}
На сторонах $AB$ и $BC$ параллелограмма $ABCD$ выбраны точки $A_1$
и $C_1$ соответственно. Отрезки $AC_1$ и $CA_1$ пересекаются в точке $P$. Описанные окружности треугольников $AA_1P$ и $CC_1P$ вторично пересекаются
в точке $Q$, лежащей внутри треугольника $ACD$. Докажите, что $\angle PDA =\angle QBA$.
\end{task}
\begin{task}
В некоторых клетках доски $10\times 10$ поставили $k$ ладей, и затем отметили все клетки, которые бьет хотя бы одна ладья (считается, что ладья бьет клетку, на которой стоит). При каком наибольшем $k$ может оказаться, что после удаления с доски любой ладьи хотя бы одна отмеченная
клетка окажется не под боем?
\end{task}

\end{document}
\begin{definition}
\end{definition}
\begin{task}
\end{task}
\begin{axiom}
\end{axiom}
\begin{assertion}
\end{assertion}
\begin{exercise}
\end{exercise}
