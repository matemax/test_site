\centerline{\bf Симедиана и центр поворотной гомотетии. 15 апреля}

\begin{definition}
Пусть $M$~--- середина стороны $BC$ треугольника $ABC$, а $N$~--- такая точка
на стороне $BC$, что $\angle BAM=\angle CAN$. Тогда $AN$ называется {\it симедианой}
треугольника $ABC$. 
\end{definition}

\begin{statement}
Пусть $\Gamma$~--- описанная окружность треугольника $ABC$. Пусть касательные к $\Gamma$
в точках $B$ и $C$ пересекаются в точке $D$. Тогда прямая $AD$ содержит симедиану треугольника $ABC$.
\end{statement}


\q1. Внутри равнобедренного треугольника $ABC$ ($AC=BC$) отмечена точка $P$ такая,
что $\angle PAB=\angle PBC$. Докажите, что если $M$~--- середина отрезка $AB$,
то $\angle APM+\angle BPC=180^\circ$.

\q2. Через точки $A$ и $B$ проведены две окружности, общая касательная к которым
пересекает их в точках $P$ и $Q$. Пусть касательные в точках $P$ и $Q$ 
к описанной окружности треугольника $APQ$ пересекаются в точке $S$; 
пусть $H$~--- точка, симметричная точке $B$ относительно прямой $PQ$.
Докажите, что точки $A$, $S$ и $H$ лежат на одной прямой. 

\medskip

\q3. Дан треугольник $ABC$. Пусть $X$~--- центр поворотной гомотетии, переводящей
направленный отрезок $BA$ в направленный отрезок $AC$. Докажите, что $AX$ 
содержит симедиану треугольника $ABC$.

\q4. Пусть $G$~--- центр масс треугольника $ABC$, точка $P$ лежит на отрезке $BC$.
Точки $Q$ и $R$ на сторонах  $AC$ и $AB$ соответственно таковы,
что $PQ\parallel AB$ и $PR\parallel AC$. Докажите, что описанные окружности
треугольника $AQR$ проходят через фиксированную точку $X$, не зависящую от точки $P$.

\q5. Пусть $ABC$~--- остроугольный неравнобердренный треугольник, а $M$, $N$ и $P$~---
середины сторон $BC$, $CA$ и $AB$ соответственно. Пусть серединные
перпендикуляры к отрезкам $AB$ и $AC$ пересекают луч $AM$ в точках $D$ и $E$ соответственно,
а прямые $BD$ и $CE$ пересекаются в точке $F$, лежащей внутри треугольника $ABC$.
Докажите, что точки $A$, $N$, $F$ и $P$ лежат на одной окружности.

\q6. Окружности $\omega_1$ и $\omega_2$ с центрами в точках $O_1$ и $O_2$ соответственно
проходят через точку $A$. Прямая $\ell$ касается $\omega_1$ и $\omega_2$ в точках
$B$ и $C$ соответственно. Пусть $O_3$~--- центр описанной окружности треугольника
$ABC$. Точка $D$ такова, что $A$~--- середина отрезка $O_3D$; $M$~--- середина $O_1O_2$. 
Докажите, что $\angle O_1DM=\angle O_2DA$.

 
