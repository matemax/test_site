\documentclass[14pt,DIV14]{scrartcl}
\usepackage[cp1251]{inputenc}
\usepackage[russian]{babel}
%\usepackage[utf8]{inputenc}
\usepackage{sheet}
\usepackage{multicol}
 \usepackage{epigraph}
  \usepackage{graphicx}
%\usepackage{xfrac}
\usepackage{a4wide}
\pagestyle{empty}
\usepackage{tikz}
\usepackage{graphicx,epstopdf}
\def\mod_#1 {\mathrel{\mathop{\equiv}\limits_{#1}}}
\newtheorem{theorem}{Теорема}
\evensidemargin=-1cm \oddsidemargin=-1cm \textwidth=18cm
\topmargin=-2cm \textheight=26cm

\begin{document}
\sheet{Малый механико-математический факультет, 13-02.}{Многочлены Чебышёва
}{8 апреля 2017 9-11 класс}

\begin{definition}
\end{definition}

\begin{task}
Пусть $T_n(x)=\cos(n\arccos x)$.  
\sub Найдите $T_n$ при $n=0,1,2,3$.

\sub Докажите, что $T_{n+1}(x) = 2T_n(x)-T_{n-1}(x)$.

\sub Докажите, что $T_n(x)$~--- многочлен степени~$n$ и найдите его старший коэффициент. Выведите рекуррентную формулу.

\sub Найдите все его корни и экстремумы.
\end{task}

\begin{task}
Докажите, что $T_n(x)$ при чётных $n$ является чётной функцией, а при нечётных~--- нечётной.
\end{task}

\begin{definition}
Многочлен $T_n(x)$ называется \emph{многочленом Чебышёва}.
\end{definition}

\begin{definition}
Величина $\max\limits_{x\in[a,b]}|f(x)|$ называется {\it уклонением от нуля} многочлена $f(x)$ 
на отрезке $[a,b]$.
\end{definition}

\begin{task}\sub Найдите уклонение многочлена Чебышёва на отрезке $[-1;1]$.

\sub Докажите,что  $\max\limits_{x\in[-1,1]}|f(x)|\ge{1\over 2^{n-1}}$.

\sub Докажите, что если $\max\limits_{x\in[-1,1]}|f(x)|={1\over 2^{n-1}}$,
то $f(x)={1\over2^{n-1}}T_n(x)$.
\end{task}

\begin{remark}
Таким образом, многочлен ${1\over2^{n-1}}T_n(x)$ является наименее уклоняющимся от нуля среди всех
унитарных многочленов степени~$n$.
\end{remark}

\begin{task}
\sub Докажите, что $T_m(T_n(x))=T_n(T_m(x))$.

\sub ~Пусть $z\in\mathbb C$, $|z|=1$. Вычислите $T_n\left({z+z^{-1}\over2}\right)$.

\sub Найдите $T_n(\sin \alpha)$.
\end{task}
\begin{task}
Докажите, что $P_n(x)=2T_n({x\over2})$ также является многочленом с целыми коэффициентами.

\sub Докажите, что если $\alpha\in\mathbb{Q}$ и $\cos(\alpha\pi)\in\mathbb{Q}$, то $\cos(\alpha\pi)\in\{0,\pm1,\pm{1\over2}\}$.

\sub Докажите, что при $n>4$ не существует правильного $n$-угольника с вершинами в узлах клечатой сетки.
\end{task}
\begin{task}
Известно, что $\sin \alpha = 3/5$. Докажите, что $\sin 25\alpha$ имеет вид
$\dfrac{n}{5^{25}}$ , где $n$~--- целое, не делящееся на 5.
\end{task}
\end{document}
\begin{definition}
\end{definition}
\begin{task}
\end{task}
\begin{axiom}
\end{axiom}
\begin{assertion}
\end{assertion}
\begin{exercise}
\end{exercise}
