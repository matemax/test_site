\centerline{\bf Поворотная гомотетия. 8 апреля}

\begin{definition}
{\it Поворотной гомотетией} называется композиция поворота и гомотетии с общим центром. 
\end{definition}

\q1. Поворотная гомотетия с центром в точке $O$ переводит точки 
$A$ и $B$ в точки $C$ и $D$ соответственно. Докажите, что существует 
поворотная гомотетия с центром в точке $O$, переводящая $A$ в $B$ и $C$ в $D$.

\q2. Даны различные точки $A$, $B$, $C$, $D$. Пусть $X$ --- точка пересечения 
прямых $AC$ и $BD$, а $O$ --- вторая точка пересечения окружностей $ABX$ и $CDX$.\\
a) Докажите, что существует поворотная гомотетия переводящая направленный 
отрезок $AB$ в направленный отрезок $CD$, причём её центр это точка $O$.\\
b) Докажите, что такая поворотная гомотетия единственна.

\q3. Докажите существование точки Микеля через задачи 1 и 2.

\q4. На сторонах $AD$ и $BC$ четырёхугольника $ABCD$ отмечены точки 
$E$ и $F$ соответственно так, что $AE/ED=BF/FC$. Луч $EF$ пересекает 
прямые $AB$ и $CD$ в точках $S$ и $T$ соответственно. Докажите, что
описанные окружности треугольников $SAE$, $SBF$, $TCF$ и $TDE$ имеют общую точку.

\q5. Четырёхугольник $ABCD$ вписан в окружность с центром $O$. 
Пусть $P$ --- пересечение диагоналей $AC$ и $BD$, а $Q$~--- 
вторая точка пересечения окружностей $ABP$ и $CDP$. Докажите, что $\angle OQP=90^\circ$.

\q6. В четырёхугольнике $ABCD$ стороны $BC$ и $AD$ равны, но не параллельны. 
На этих сторонах отметили точки $E$ и $F$ соответственно так, что $BE=FD$. 
Прямые $AC$ и $BD$ пересекаются в точке $P$, прямые $EF$ и $BD$~--- в точке
$Q$, прямые $EF$ и $AC$~--- в точке $R$. Докажите, что описанная окружность треугольника $PQR$
проходит через некоторую точку $S\neq P$, не зависящую от точек $E$ и $F$.

\q7. В выпуклом пятиугольнике $ABCDE$ $\angle BAC=\angle CAD=\angle DAE$ и 
$\angle CBA=\angle DCA=\angle EDA$. Диагонали $BD$ и $CE$ пересекаются в точке~$P$.
Докажите, что прямая $AP$ делит сторону $CD$ пополам.



 
