\centerline{\bf Среднее значение функции. 15 октября}

\vskip -.5cm

\phantom{1}\hfill {\it Синий косяк.}\\
\phantom{1}\hfill {\it Косяк~--- синий.}

\q1. Вычислите a) $\int_0^{2\pi} \cos x\ \mathrm{d}x$;\quad b) $\int_0^{2\pi} |\cos x|\ \mathrm{d}x$.

\q2. На плоскости дан некоторый отрезок длины $a$. Обозначим через $\ell_a(\alpha)$
длину проекции отрезка $a$ на прямую, образующую угол $\alpha$ с заданным направлением.
Найдите $\int_0^{2\pi} \ell_a(\alpha)\ \mathrm{d}\alpha$.

\q3. На плоскости даны два выпуклых многоугольника $\Phi_1$ и $\Phi_2$,
причём $\Phi_2$ лежит строго внутри $\Phi_1$. Пользуясь результатом задачи {\bf 2.}
докажите, что периметр $\Phi_2$ меньше периметра $\Phi_1$.

\q4. a) Докажите, что если длины всех сторон и диагоналей выпуклого многоугольника
не превосходят $d$, то его периметр не превосходит $\pi d$.\\
b) Докажите, что константа $\pi$ в пункте a) явялется точной.

\q5. Докажите, что периметр выпуклой оболочки любой замкнутой ломанной не превосходит
длины этой ломанной.

\q6. На плоскости даны векторы $\vec{a}$, $\vec{b}$, $\vec{c}$, $\vec{d}$, сумма которых равна $\vec{0}$.
Докажите, что $|\vec{a}|+|\vec{b}|+|\vec{c}|+|\vec{d}|\geqslant |\vec{a}+\vec{d}|+|\vec{b}+\vec{d}|+|\vec{c}+\vec{d}|$.

\q7. a) На плоскости даны векторы $\vec{a_1}$, $\vec{a_2}$, \ldots, $\vec{a_n}$, сумма длин которых
равна 1. Докажите, что среди них можно выбрать несколько векторов, длина суммы которых
не меньше $1/\pi$.\\
b) Докажите, что константа $1/\pi$ в пункте a) является точной.

% \q7 (Игла Бюффона). Иголку длиной 10 см случайно бросают на разлинованную бумагу, 
% где расстояние между соседними линиями тоже 10 см. Какова вероятность того,
% что упавшая игла пересекает одну из прямых? 
% 
% 
% 
% \medskip
% \centerline{\it Задачи для самостоятельного решения}
% \medskip
% 
% \q8. На круглом столе как-то лежат 50 правильно идущих круглых часов. Докажите, что в 
% некоторый момент сумма расстояний от центра стола до концов минутных 
% стрелок окажется больше суммы расстояний от центра стола до центров
% часов.
% 
% \q9. В московском метро можно провозить коробки, у которых сумма измерений (длины, ширины и высоты)
% не превосходит некоторой границы. Возникает вопрос: можно ли перехитрить правила, поместив одну 
% коробку внутрь другой? Другими словами, пусть один прямоугольный параллелепипед целиком содержится внутри
% другого. Может ли сумма измерений внутреннего быть больше суммы измерений внешнего?
% 
