\centerline{\bf Разнобойчик от А.В. 22 октября}

\q1. Выпуклый многоугольник обладает следующим свойством: если все прямые, на которых лежат
его стороны, параллельно перенести на расстояние 1 во внешнюю сторону, то полученные прямые
образуют многоугольник, подобный исходному, причём параллельные стороны окажутся
пропорциональными. Докажите, что в данный многоугольник можно вписать окружность.
% Московская Математическая Олимпиада, 1974

\q2. В треугольной пирамиде $ABCD$ все плоские углы при вершинах~--- не прямые, 
а точки пересечения высот в треугольниках $ABC$, $ABD$, $ACD$ лежат на одной прямой. 
Докажите, что центр описанной сферы пирамиды лежит в плоскости, 
проходящей через середины ребер $AB$, $AC$, $AD$.
% Всероссийская Олимпиада Школьников, финал, 2009 

\q3. У тетраэдра $ABCD$ все двугранные углы острые, а противоположные рёбра попарно
равны. Найти сумму косинусов всех двугранных углов тетраэдра.
% Московская Математическая Олимпиада, 19мохнатый

\q4. Дана замкнутая пространственная ломаная с вершинами $A_1$, $A_2$, \ldots, $A_n$,
причём каждое звено пересекает фиксированную сферу в двух точках, а все вершины ломаной
лежат вне сферы. Эти точки делят ломаную на $3n$ отрезков. 
Известно, что отрезки, прилегающие к вершине $A_1$, равны между собой. 
То же самое верно и для вершин $A_2$, $A_3$, \ldots, $A_{n-1}$. 
Докажите, что отрезки, прилегающие к вершине $A_n$, также равны между собой.
% Московская Математическая Олимпиада, 1971

% \q5. В колбе находится колония из $n$ бактерий. В какой-то 
% момент внутрь колбы попадает вирус. В первую минуту вирус уничтожает одну бактерию, и сразу же после этого
% и вирус, и оставшиеся бактерии делятся пополам. Во вторую минуту новые два вируса уничтожают две бактерии, а
% затем и вирусы, и оставшиеся бактерии снова делятся пополам, и т.д. Наступит ли такой момент времени, когда
% не останется ни одной бактерии?
% % Московская Математическая Олимпиада, 1971

\q5. Можно ли каждую сторону квадрата так разделить на 100 частей, чтобы из полученных 400 отрезков нельзя 
было бы составить контура никакого прямоугольника, отличного от исходного квадрата?
% Московская Математическая Олимпиада, 1971

\q6. В пространстве даны точка $O$ и $n$ попарно непараллельных прямых. Точка $O$ ортогонально проектируется на
все данные прямые. Каждая из получившихся точек снова проектируется на все данные прямые и т. д. Существует
ли шар, содержащий все точки, которые могут быть получены таким образом?
% Московская Математическая Олимпиада, 1971