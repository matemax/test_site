\documentclass[a4paper,14pt]{extarticle}
% Русский язык %
\usepackage[utf8]{inputenc}
\usepackage[russian]{babel}

% Для поиска и копирования текста
\usepackage{cmap}

% AMS %
\usepackage{amssymb}
\usepackage{amsmath}
\usepackage{amsthm}

% Для правильного зачеркивания %
\usepackage{centernot}
\usepackage{cancel}

% Для импорта графики %
\usepackage{wrapfig}
\usepackage{graphicx}


\renewcommand{\geq}{\geqslant}
\renewcommand{\leq}{\leqslant}
\newcommand{\divby}{\mathrel{\smash{\lower.5ex\hbox{$\,\vdots\,$}}}}
\newcommand{\ndivby}{\centernot\divby}
\newcommand{\imply}{\ensuremath{\Rightarrow} }

\def\mod_#1 {\mathrel{\mathop{\equiv}\limits_{#1}}}
\def\q#1. {\medskip\noindent\phantom{1}{\bf#1.} }

\theoremstyle{plain} 
\newtheorem{theorem}{Теорема}
\newtheorem{hypothesis}{Гипотеза}
\newtheorem{corollary}{Следствие}
\newtheorem{statement}{Утверждение}
\newtheorem{lemma}{Лемма}

\theoremstyle{definition} 
\newtheorem{definition}{Определение}
\newtheorem{example}{Пример}
\newtheorem{task}{Упражнение}
\newtheorem{problem}{Задача}

\theoremstyle{remark} 
\newtheorem*{remark}{\bf Замечание}

\pagestyle{empty}

\hoffset=-1in %всю страницы влево
\voffset=-1in %всю страницу вверх

\headheight=0in %высота колонтитула
\headsep=0in %отступ от колонтитула

\textwidth=190mm %ширина листа
\textheight=277mm %высота листа

\oddsidemargin=10mm %отступ от левого края
\topmargin=10mm %отступ от верхнего края




\begin{document}

\centerline{{\bf Десятичные дроби}}
\centerline{{\bf Теория}}
\medskip

Целью нашего разговора будет как можно больше понять про десятичные представление рациональных чисел $\frac{m}{n}$.
Во-первых, сразу откажемся от случая $m\geqslant n$, а также будем считать дробь нескократимой: $(m,n)=1$. 

Во-вторых, нас интересуют лишь бесконечные десятичные дроби. Несложно проверить,
что дробь будет конечной тогда и только тогда, когда $n=2^x\cdot 5^y$.
Действительно, если $\frac{m}{n}=\overline{0{,}a_1a_2\ldots a_t}=\frac{\overline{a_1a_2\ldots a_t}}{10^t}$,
то $10^t\divby n$, откуда $n$~--- произведение степени 2 на степень 5. С другой стороны, если $n=2^x5^y$,
то 
\[
\frac{m}{n}=\frac{m\cdot 2^{\max(x,y)-x}5^{\max(x,y)-y}}{10^{\max(x,y)}}=\overline{0{,}\ldots} 
\]
конечная десятичная дробь.
В дальнейшем считаем, что $n\neq 2^x\cdot 5^y$.


\begin{example}
Рассмотрим дробь $\frac{70}{101}$. Как представить её в виде бесконечной десятичной дроби? Попробуем поделить в столбик!
\ldots * делим в столбик * \ldots
Итак, мы зациклились, поэтому $\frac{70}{101}=0{,}69306930\ldots=0{,}(6930)$.
\end{example}

\begin{example}
Аналогично получаем $\frac{1}{7}=0{,}(142857)$.
\end{example}

Понятно, что все те же рассуждения можно провести для произвольной дроби $\frac{m}{n}$.
Так что мы доказали следующую теоремы.

\begin{theorem}
Для любых натуральных $m<n$, $(m,n)=1$, $n\neq 2^x\cdot 5^y$, десятичная запись дроби
$\frac{m}{n}$ является бесконечной периодической десятичной дробью. 
\end{theorem}

Более того, как только повторится остаток, который уже встречался ранее, мы войдём в период.
Однако остатков при делении на $n$ у нас ровно $n$. Среди остатков, которые нам встречались в ходе деления в столбик,
не может получится остаток 0. Значит, различных остатков, которые нам могут встретится, не более $n-1$.

\begin{theorem}
В условиях предыдущей теоремы, длина периода дроби $\frac{m}{n}$ не превосходит $n-1$.
Более того, сумма длин периода и предпериода не превосходит $n-1$.
\end{theorem}

Зададимся обратным вопросом --- как из бесконечной чисто периодической дроби получить 
дробь обычную? Т.е. пусть $\alpha=0{,}(A)_k$. Чему тогда равно $\alpha$?
\begin{eqnarray*}
\alpha &=&0{,}(A)_k \\
10^k\alpha &=&A{,}(A)_k
\end{eqnarray*}
Вычитая из второго равенства первое, получаем $(10^k-1)\alpha=A$, т.е. $\alpha=\frac{A}{10^k-1}$.

Можно пойти по другому пути: 
\[
\alpha=A\cdot 10^{-k}+A\cdot 10^{-2k}+\ldots=A\cdot (10^{-k}+10^{-2k}+\ldots)=A\cdot \frac{10^{-k}}{1-10^{-k}}=\frac{A}{10^k-1}. 
\]

Хорошо, а если в нашей дроби имеется предпериод?

\begin{example}
Как представить в виде обыкновенной дроби число $\alpha=0{,}05(61)$?
Делаем аналогично:
\begin{eqnarray*}
10^2\alpha &=& 5{,}(61) \\
10^4\alpha &=& 561{,}(61),
\end{eqnarray*}
откуда $\alpha=\frac{561-5}{10^4-10^2}$.
\end{example}

Обобщая это рассуждение, получаем, что если $\alpha=0{,}B_m(A)_k$, то $\alpha=\frac{\overline{BA}-B}{10^m(10^k-1)}$.

Теперь мы готовы ответить на следующий вопрос: глядя на дробь $\frac{m}{n}$, как сказать, будет ли в её десятичном
представлении предпериод или нет?

\begin{theorem}
Десятичная запись дроби $\frac{m}{n}$ будет чисто периодической тогда и только тогда, когда $\text{НОД}(n,10)=1$.
\end{theorem}

\begin{proof}
Если дробь является чисто периодической $\frac{m}{n}=0{,}(A)_k=\frac{A}{10^k-1}$, то $10^k-1\divby n$,
т.е. $n$ не делится ни на 2, ни на 5.

С другой стороны, если предпериод есть, то $\frac{m}{n}=0{,}B_\ell(A)_k=\frac{\overline{BA}-B}{10^\ell(10^k-1)}$.
Если $\overline{BA}-B$ не делится на $10^\ell$, то в $n$ останется или 2, или 5. Предположим противное и
пусть $\overline{BA}-B$ делится на $10^\ell$. Но тогда $\frac{m}{n}=\frac{C}{10^k-1}=0{,}(C)_k$~--- чисто
периодическая дробь, что противоречит условию. 
\end{proof}

Следующий вопрос, наверное, является самым важным. Пусть $\text{НОД}(n,10)=1$. Как, глядя на дробь
$\frac{m}{n}$ узнать, чему равняется длина $k$ периода? Зная ответ, его не очень сложно проверить,
что мы оставляем в качестве упражнения.

\begin{theorem}
Пусть $\text{НОД}(n,10)=1$. Тогда длина периода десятичной записи дроби $\frac{m}{n}$ равняется наименьшему
натуральному $k$ такому, что $10^k-1$ делится на $n$.
\end{theorem}

Из этой теоремы следует несколько интересных наблюдений.
Во-первых, для любого натурального $n$, $\text{НОД}(n,10)=1$ существует такое натуральное $k$,
что $10^k-1$ делится на $n$. Во-вторых, длина периода дроби $\frac{m}{n}$ \underline{не зависит}
от $m$.

Хорошо, разобрались с длиной периода. А что с предпериодом?

\begin{theorem}
Пусть $n=2^s\cdot 5^t\cdot n_1$, $\text{НОД}(n_1,10)=1$. 
Тогда длина предпериода десятичной записи дроби $\frac{m}{n}$ равняется $\max\{s,t\}$.
\end{theorem}

{\it Доказательство} остаётся в качестве упражнения.

\bigskip
\centerline{\bf Несколько интересных сюжетов}

\begin{example}
Пусть $(n,10)=1$. Рассмотрим дробь $\frac{1}{n}=0{,}(A)$. Например, $\frac{1}{7}=0{,}(142857)$.
Если умножить $\frac{1}{n}$ на 10, то из $A$ <<вперёд>> вылезет одна цифра, а период циклически сдвинется:
$\frac{10}{n}=a{,}(A')$. Тогда $\frac{10\,\mathrm{mod}\,n}{n}=\frac{t}{n}=0{,}(A')$.

С другой стороны, дробь $\frac{t}{n}$ можно получить из $\frac{1}{n}$ простым умножением на $t$.
При этом период также умножится на $t$, т.е. $A'=t\cdot A$. Вместо числа 10 можно было умножить
исходное выражение на 100, 1000, \ldots. Тогда период циклически сдвигался бы на 2, на 3, \ldots цифры

Итак, мы получили, что любое число, получаемое циклическим сдвигом из $A$ делится на $A$.
Например, каждое из чисел 428571, 285714, 857142, 571428, 714285 делится на 142857.
\end{example}

\begin{example}
Пусть $p>5$~--- простое такое, что длина периода $A$ дроби $\frac{m}{p}$ равняется $2k$ для некоторого натурального $k$.
Например, $\frac{1}{7}=0{,}(142857)$. Разделим число $A$ на две части по $k$ цифр в каждой: $A=\overline{A_1A_2}$.
Тогда $A_1+A_2=99\ldots 9$. Например, $142+857=999$.

Действительно, $\frac{m}{p}=0{,}(A_1A_2)$, $\frac{10^km}{p}=A_1{,}(A_2A_1)$.
Заметим, что $\frac{m}{p}+\frac{10^km}{p}=m\cdot \frac{10^k+1}{p}$.
Учитывая теорему 4, имеем $10^{2k-1}-1=(10^k-1)(10^k+1)\divby p$, причём в силу минимальности $2k$ 
получаем, то $10^k-1\ndivby p$, т.е. $10^k+1\divby p$.
Следовательно, сумма чисел $\frac{m}{p}+\frac{10^km}{p}$ является целым числом!

Несложно проверить, что это и означаем, что сумма цифр в каждом разряде после запятой этих чисел
равняется 9, т.е. $A_1+A_2=99\ldots 9$.
\end{example}

\begin{example}
Пусть $p$~--- очень большое простое число (например, большее $10^{10}$), а длина
периода дроби $\frac{1}{p}$ равняется $p-1$, т.е. $10$ принадлежит показателю $p-1$ по модулю $p$.

Ещё раз аккуратно посмотрим на процедуру деления в столбик 1 на $p$:
\[
\left\{\begin{array}{rcccl}
        10 & = & pq_1 & + & r_1 \\
        10r_1 & = & pq_2 & + & r_2 \\
        10r_2 & = & pq_3 & + & r_3 \\
        \ldots 
        \end{array}
\right.
\]
При этом $\overline{q_1q_2\ldots q_{p-1}}$~--- период дроби $\frac{1}{p}$.

Поскольку длина периода равняется $p-1$, среди чисел 1, $r_1$, $r_2$, \ldots, $r_{p-2}$
нет повторяющихся (иначе зацикливание произошло бы раньше), т.е. они образуют приведённую систему вычетов по модулю $p$. 

А как найти $\overline{q_{k=1}q_{k+2}\ldots q_{k+10}}$? Посмотрим на равенство $10r_k=pq_{k+1}+r_{k+1}$.
Умножим его на 10: 
\[
10^2r_k=p\cdot 10q_{k+1}+10r_{k+1}=p\cdot (10q_{k+1}+q_{k+2})+r_{k+2}=p\cdot \overline{q_{k+1}q_{k+2}}+r_{k+2}. 
\]
Если мы ещё раз умножим полученное равенство на 10, получим 
\[
10^3r_k=p\cdot \overline{q_{k+1}q_{k+2}q_{k+3}}+r_{k+3}.
\]
Продолжая аналогично, получим, что 
\[
10^{10}r_k=p\cdot \overline{q_{k+1}q_{k+2}\ldots q_{k+10}}+r_{k+10}.
\]

Выпишем все эти равенства:
\[
\left\{\begin{array}{rcccl}
        10^{10} & = & p\cdot \overline{q_1q_2\ldots q_{10}} & + & r_{10} \\
        10^{10}r_1 & = & p\cdot \overline{q_2q_3\ldots q_{11}} & + & r_{11} \\
        \ldots 
       \end{array}
\right..
\]
В левой части встречают все из чисел $10^{10}$, $2\cdot 10^{10}$, \ldots, $(p-1)\cdot 10^{10}$.
\[
\left\{\begin{array}{rcccl}
        10^{10} & = & p\cdot Q_1 & + & R_1 \\
        2\cdot 10^{10}r_1 & = & p\cdot Q_2 & + & R_2 \\
        \ldots 
       \end{array}
\right.. 
\]
Заметим, что левая часть каждый раз увеличивается на $10^{10}$, а поскольку $p>10^{10}$,
то $Q$ увеличивается каждый раз не более чем на 1. Следовательно,
среди чисел $Q_1$, $Q_2$,~\ldots, $Q_{p-1}$ встретятся все не более чем 10-значные числа.
\end{example}

\bigskip
\centerline{\bf Упражнения}

\q1. Пусть $p$~--- простое число, а $m<p$~--- натуральное число. Оказалось, что
$\frac{m}{p}=0{,}(A)_{3k}$. Докажите, что если разбить число $A$ на три числа $A_1$, $A_2$, $A_3$ по $k$ цифр:
$A=\overline{A_1A_2A_3}$, то $A_1+A_2+A_3=99\ldots 9$ или $A_1+A_2+A_3=2\cdot 99\ldots 9$ ($k$ девяток).

\q2. Докажите, что для каждого положительного действительного $\alpha$ найдутся такие
действительные числа $\beta_1$, $\beta_2$, \ldots, $\beta_9$, в десятичной записи которых встречаются только цифры 0 и 7,
что выполнено равенство $\alpha=\beta_1+\beta_2+\ldots+\beta_9$.

\q3. Пусть $p$, $q>5$~--- простые числа. Длины периодов дробей $\frac{1}{p}$ и $\frac{1}{q}$ равны $a$ и $b$ соответственно.
Найдите длину периода дроби $\frac{1}{pq}$. 






\end{document}
