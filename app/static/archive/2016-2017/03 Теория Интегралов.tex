\centerline{{\bf Интегралы: теория. 8 октября}}
\medskip

Напомним некоторые определения и доказанные утверждения. 

\begin{definition}
{\it Разбиением} $P$ отрезка $[a,b]$, $a<b$, называется конечная система точек
$a=x_0<x_1<\ldots<x_n=b$. 

Отрезки $[x_{i-1},x_i]$ называются {\it отрезками разбиения}.

Максимум из длин отрезков разбиения называется {\it диаметром разбиения} ({\it параметром разбиения}).
\end{definition}

\begin{definition}
Говорят, что имеется {\it отмеченное разбиение} $(P,\xi)$ отрезка $[a,b]$,
если имеется разбиение $P$ отрезка $[a,b]$ и в каждом из отрезков $[x_{i-1},x_i]$ разбиения~$P$
выбрано по точке $\xi_i\in [x_{i-1},x_i]$.
\end{definition}

\begin{definition}
Говорят, что число $I$ является {\it определённым интегралом (Римана)} от функции $f$
на отрезке $[a,b]$, если для любого $\varepsilon>0$ найдётся число $\delta>0$ такое,
что для любого отмеченного разбиения $(P,\xi)$ отрезка $[a,b]$, диаметр которого меньше $\delta$,
выполнено
\[
\left|I-\sum_{i=1}^nf(\xi_i)\cdot |x_i-x_{i-1}|\right|<\varepsilon.
\]

Если для функции $f(x)$ существует определённый интеграл (Римана) от функции $f(x)$ на отрезке $[a,b]$,
то говорят, что она {\it интегрируема (по Риману) на отрезке $[a,b]$}.
\end{definition}

\begin{statement}
Если функция $f(x)$ интегрируема на отрезке $[a,b]$, то она ограничена на этом отрезке.
\end{statement}

\begin{definition}
Пусть $f(x)$~--- некоторая ограниченная на отрезке $[a,b]$ функция,
$P$~--- некоторое разбиение отрезка $[a,b]$. {\it Нижней и верхней интегральной суммой}
или {\it суммами Дарбу} называется выражение
\[
s=\sum_{i=1}^n \inf_{x\in [x_{i-1},x_i]} f(x)\cdot |x_i-x_{i-1}|\text{ и }
S=\sum_{i=1}^n \sup_{x\in [x_{i-1},x_i]} f(x)\cdot |x_i-x_{i-1}|
\]
соответственно.
\end{definition}

\begin{theorem}
Ограниченная на отрезке $[a,b]$ функция $f(x)$ интегрируема (по Риману) на нём тогда и только тогда,
когда для любого $\varepsilon>0$ найдётся число $\delta>0$ такое, что для любого разбиения $P$
диаметром меньше $\delta$ выполнено $|S-s|<\varepsilon$.
\end{theorem}



