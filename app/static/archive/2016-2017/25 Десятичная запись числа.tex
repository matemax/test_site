\centerline{\bf Десятичные дроби. 18 марта}

\q1. Пусть $p$, $q>5$~--- простые числа. Длины периодов дробей $\frac{1}{p}$ и $\frac{1}{q}$ равны $a$ и $b$ соответственно.
Найдите длину периода дроби $\frac{1}{pq}$. 

\q2. Докажите, что для каждого положительного действительного $\alpha$ найдутся такие
действительные числа $\beta_1$, $\beta_2$, \ldots, $\beta_9$, в десятичной записи которых встречаются только цифры 0 и 7,
что выполнено равенство $\alpha=\beta_1+\beta_2+\ldots+\beta_9$.

\q3. Пусть $(n,10)=1$ и $\frac{1}{n}=0,(A)$. Докажите, что любое число, которое
получается из $A$ циклическим сдвигом, делится на $A$. \\{\it Например, $\frac{1}{7}=0{,}(142857)$,
а каждое из чисел 428571, 285714, 857142, 571428, 714285 делится на 142857.}

\q4. Пусть $p>100$~--- простое число, а $m<p$~--- натуральное число.
a) Пусть $p>5$~--- простое такое, что длина периода $A$ дроби $\frac{m}{p}$ равняется $2k$ для некоторого натурального $k$.
Разделим число $A$ на две части по $k$ цифр в каждой: $A=\overline{A_1A_2}$.
Тогда $A_1+A_2=99\ldots 9$. {\it Например, $\frac{1}{7}=0{,}(142857)$, а $142+857=999$.} \\
b) Поисследуйте ситуацию, когда длина периода делится на $t$ и мы делим
период на $t$ частей. 

\q5. Пусть $p$~--- очень большое простое число (например, большее $10^{10}$), а длина
периода дроби $\frac{1}{p}$ равняется $p-1$, т.е. $10$ принадлежит показателю $p-1$ по модулю $p$.
\footnote{Кстати, конечно ли множество таких простых чисел, неизвестно...}
Докажите, что для любого десятизначного числа $K$ в записи дроби $\frac{1}{p}$
можно выделить десять подряд идущих цифр, которые образуют число $K$.

\q6. Дима посчитал факториалы всех натуральных чисел от 80 до 99,
нашёл числа, обратные к ним, и напечатал получившиеся десятичные
дроби на 20 бесконечных ленточках (например,
на последней ленточке было напечатано число 
$\frac{1}{99!}=0{,}\underbrace{00\ldots 0}_\text{155 нулей}10715\ldots$).
Саша хочет вырезать из одной ленточки кусок, на котором записано $N$
цифр подряд и нет запятой. При каком наибольшем $N$ он сможет это
сделать так, чтобы Дима не смог определить по этому куску,
какую ленточку испортил Саша?
