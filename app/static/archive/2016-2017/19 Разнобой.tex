\centerline{\bf Алгебраический разнобой. 28 января}

\q1. На доске записаны несколько чисел. Известно, что квадрат любого записанного числа больше произведения 
любых двух других записанных чисел. Какое наибольшее количество чисел могло быть записано на доске?

\q2. Ненулевые числа $a$ и $b$ таковы, что уравнение $a(x-a)^2+b(x-b)^2=0$ имеет единственное решение. 
Докажите, что $|a|=|b|$.

\q3. Косинусы углов одного треугольника соответственно равны синусам углов другого треугольника. 
Найдите наибольший из шести углов обоих треугольников.

\q4. Решите в действительных числах уравнение 
\[
(5x+3)(x^2+x+1)(x^2+2x+3)=1001. 
\]

\q5. Пусть $a>0$. Рассмотрим последовательность, заданную следующим образом: $a_1=0$, $a_{n+1}=\sqrt{a_n+a}$
для всякого натурального $n$. Докажите, что в этой последовательности встретится бесконечно много иррациональных чисел.

\q6. Пусть $a$~--- вещественное число. Зададим последовательность $x_1$, $x_2$, $x_3$~\ldots 
условиями $x_1=1$ и $ax_n=x_1+x_2+\ldots+x_{n+1}$ при всех $n\geqslant 1$. 
Найдите наименьшее $a$, при котором все члены этой последовательности неотрицательны.

