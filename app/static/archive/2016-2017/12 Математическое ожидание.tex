\centerline{\bf Математическое ожидание. 17 декабря}

\begin{definition}
Пусть $(\Omega,\mathcal{A},P)$~--- вероятностное пространство.
{\it Случайной величиной} называется произвольная функция $\xi\colon \Omega\to\mathbb{R}$.
($\xi$ читается как <<кси>>.)
\end{definition}

\begin{definition}
Пусть случайная величина $\xi$ принимает конечное множество значений $a_1$, $a_2$, \ldots, $a_s$.
{\it Математическим ожиданием $\xi$} называется число $\mathbb{E}\xi=\sum\limits_{k=1}^s a_k\cdot P(\xi=a_k)$.
\end{definition}

\q1. Пусть $\Omega=\{x_1,x_2,\ldots, x_n\}$, $P(x_i)=p_i$. Докажите, что
$\mathbb{E}\xi=\sum\limits_{k=1}^n p_k\cdot \xi(x_k)$.

\q2 (Линейность математического ожидания).
Пусть $\xi$, $\eta$ ($\eta$ читается как <<эта>>)~--- две случаные величины 
(на одном и том же вероятностном пространстве), $c$~--- некоторая постоянная. 

Докажите, что $\mathbb{E}(\xi+\eta)=\mathbb{E}\xi+\mathbb{E}\eta$ и
$\mathbb{E}(c\cdot \xi)=c\cdot \mathbb{E}\xi$.

\medskip

\q3. Честную монету подбрасывают 100 раз подряд. Найдите математическое ожидание числа орлов.

\q4. По веревочной лестнице в ужасную грозу поднимаются 7 гномов. Если случится гром, то каждый 
гном с испугу может упасть вниз с вероятностью $p=0{,}2$. Если гном падает, то он увлекает за собой 
всех гномов, которые под ним, и они тоже падают. Вдруг раздался гром. Сколько гномов следует ожидать Белоснежке внизу? 

% \q5. В кругу стояли $n$ детей. Внезапно очень сонный преподаватель случайным образом раздал им бейджики. 
% Найдите математическое ожидание детей, получивший свой бейджик.
% 
% \q6. В кругу стояли 2016 детей. Внезапно каждый из них дал леща своему правому или левому соседу. 
% Найдите математическое ожидание числа детей, оставшихся без леща.

\begin{definition}
Пусть $p\in[0,1]$, $n\in \mathbb{N}$. Под {\it случайным графом} $G(n,p)$ мы понимаем вероятностное пространство, элементами
которого являются графы на (одних и тех же) $n$ вершинах, при этом вероятность графа с $k$ рёбрами равна $p^k(1-p)^{C_n^2-k}$.
\end{definition}

\q5. Рассмотрим случайный граф $G(n,p)$. Для данных $n$ и $p$ найдите математическое
ожидание в этом графе количества\\
a) ребер;\\
b) треугольников;\\
c) изолированных вершин;\\
d) гамильтоновых циклов;\\
e) простых циклов длины $k$;

\q6. Докажите, что для любого $n\in\mathbb{N}$ существует полный ориентированный 
граф с $n$ вершинами и по меньшей мере $2^{1-n}\cdot n!$ гамильтоновыми путями.

\newpage


\q7. Поля шахматной доски пронумерованы естественным образом числами от 1 до 64. 
На доску случайным образом поставлено 6 ладей, которые не бьют друг друга.
Найдите математическое ожидание суммы номеров полей, занятых ладьями.

\q8. Преподаватель кружка по теории вероятностей откинулся в кресле и посмотрел на экран. 
Список записавшихся готов. Всего получилось $n$ человек. Только они пока не по алфавиту, 
а в случайном порядке, в каком они приходили на занятие. 

--- Надо отсортировать их в алфавитном порядке, -- подумал преподаватель. -- Пойду по порядку сверху вниз,
и, если нужно, буду переставлять фамилию ученика вверх в подходящее место. 
Каждую фамилию придѐтся переставить не более одного раза.

Найдите математическое ожидание числа фамилий, которые не придѐтся переставлять.



\end{document}
