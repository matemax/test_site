\documentclass[14pt,DIV14]{scrartcl}
\usepackage[cp1251]{inputenc}
\usepackage[russian]{babel}
%\usepackage[utf8]{inputenc}
\usepackage{sheet}
\usepackage{multicol}
 \usepackage{epigraph}
  \usepackage{graphicx}
%\usepackage{xfrac}
\usepackage{a4wide}
\pagestyle{empty}
\usepackage{tikz}
\usepackage{graphicx,epstopdf}
\def\mod_#1 {\mathrel{\mathop{\equiv}\limits_{#1}}}
\newtheorem{theorem}{Теорема}
\evensidemargin=-1cm \oddsidemargin=-1cm \textwidth=18cm
\topmargin=-2cm \textheight=26cm

\begin{document}
\sheet{Малый механико-математический факультет, 13-02.}{Наборы точек на плоскости
}{1 апреля 2017 9-11 класс}

\begin{task}[Теорема Сильвестра-Галлаи]
\sub Пусть даны $n$ точек на плоскости, не все точки лежат на одной прямой. Тогда найдётся прямая,
которая пройдёт ровно через две точки.\\
\sub Сформулируйте обобщение  данной теоремы на случай трёхмерного пространства,  верно ли оно?
\end{task}

\begin{task}
Предположим, что отмечены $n$ точек на плоскости,
не лежащих на одной прямой. Тогда найдется по меньшей мере $n$
прямых, соединяющих пары отмеченных точек.
\end{task}

\begin{exercise}
На плоскости отмечено  $n$ не параллельных прямых, не все из которых пересекаются в одной точке. Тогда существуют минимум $n$ точек пересечения. 
\end{exercise}

\begin{definition}
\emph{Числом скрещиваний} $cr(\overline{G})$ изображения $\overline{G}$ графа на плоскости называется число пар таких пересекающихся ребер, которые не
имеют общих вершин.\\
\emph{Числом скрещиваний} $cr(G)$ графа $G$ называется минимальное
число скрещиваний среди всех изображений графа на плоскости.
\end{definition}


\begin{task}
Докажите, что для любого графа $G$ с $n$ вершинами и $e$ рёбрами выполнено неравенство $cr(G) > e - 3n$.
\end{task}


\begin{task}
Дан граф $G$ с $n$ вершинами и $e$ рёбрами. Пусть $T'$ его реализация на плоскости с $cr(T)=ch(G)$. Рассмотрим вероятностный граф $G'$ который получается из $G$ выбором каждой вершины с вероятностью $p$ и соответствующая ему реализация $T'$, полученный из $T$.\\ 
\sub $E(cr(T'))=p^4cr(G)$.\\
\sub Дан граф $G$ с $n$ вершинами и $e$ рёбрами. Если $e > 4n$, то $cr(G) > \dfrac{e^3}{64n^2}$.\\
\sub Получите оценку на число скрещиваний для полного графа.
\end{task}

\begin{definition}
Точки на плоскости, некоторые из которых соединены отрезками называются  \emph{геометрическим графом}.
\end{definition}
\begin{task}
Пусть $P$ множество точек на плоскости из $n$ штук и $G$ геометрический граф на этих точках. Докажите, что если любые два ребра пересекаются и никакие два  ребра не совпадают по отрезку, то в $G$ не больше $n$ рёбер.
\end{task}
\begin{task}
Let $L$ be a set of $n$ non-concurrent lines in the real affine plane. Assume
that no two lines in $L$ are parallel. Then $L$ determines at least $n$ extreme intersection points.
\end{task}

\end{document}
\begin{definition}
\end{definition}
\begin{task}
\end{task}
\begin{axiom}
\end{axiom}
\begin{assertion}
\end{assertion}
\begin{exercise}
\end{exercise}
