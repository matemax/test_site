\centerline{\bf Вероятность. 10 декабря}

\q1. Имеется 5 белых и 5 чёрных шаров. Мы достаем их по одному и кладём их в ряд. В результате 
получится <<случайная>> последовательность из белых и чёрных шаров. Можно задать два вероятностных пространства:
когда все шары различны (т.е. в $\Omega$ всего $10!$ вариантов), а также когда все белые и все чёрные шары одинаковы между собой 
(т.е. $C_{10}^5$ вариантов).

Равны или нет в этих пространствах вероятности событий ``какие-то три подряд шара --- белые''?

\q2. Встретились русский математик, француз и американец. <<У меня двое детей, один из которых мальчик>>, -- 
сказал француз. <<А у меня двое детей, один из которых мальчик, родившийся до полудня>>, -- сказал американец. 
Русский математик предположил, что девочки и мальчики рождаются с вероятностью 1/2, и что 
дети рождаются до и после полудня с вероятностью 1/2. Он посчитал два числа: $p_1$ -- вероятность того, 
что другой ребенок у француза девочка и $p_2$ -- вероятность того, что второй ребенок у американца девочка.
Чему равно $p_1/p_2$? 

% \q3. Представьте, что вы стали участником игры, в которой вам нужно выбрать одну из трёх дверей. 
% За одной из дверей находится автомобиль, за двумя другими дверями --- козы. 
% Вы выбираете одну из дверей, например, номер 1, после этого ведущий, который знает, 
% где находится автомобиль, а где --- козы, открывает одну из оставшихся дверей, например, 
% номер 3, за которой находится коза. После этого он спрашивает вас --- не желаете ли вы 
% изменить свой выбор и выбрать дверь номер 2? Увеличатся ли ваши шансы выиграть автомобиль, 
% если вы примете предложение ведущего и измените свой выбор?

\q3.  В самолет, на который раскуплены все билеты, садятся $100$ 
пассажиров --- $99$ нормальных и одна безумная старушка. Первой влетает в самолёт безумная старушка
и занимает случайное место, даже не заглядывая в свой билет. Дальше пассажиры заходят в 
самолёт по одному в случайном порядке. Каждый из 
нормальных пассажиров старается сесть на своё место, но если оно уже занято, 
то он садится на произвольное из свободных. Какова вероятность того, что последний пассажир сядет на своё место?
%Ответ: бабка надвое сказала. Доказывается по индукции.





\end{document}
