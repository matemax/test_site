\documentclass[14pt]{extarticle}
\usepackage{amssymb}
\usepackage{amsmath}
\usepackage[utf8]{inputenc}
\usepackage[russian]{babel}

\pagestyle{empty}

\hoffset=-1in %всю страницы влево
\voffset=-1in %всю страницу вверх

\headheight=0in %высота колонтитула
\headsep=0in %отступ от колонтитула

\textwidth=190mm %ширина листа
\textheight=277mm %высота листа

\oddsidemargin=10mm %отступ от левого края
\topmargin=10mm %отступ от верхнего края


%\def\q#1. {\noindent\phantom{1}{\bf#1.} }

\begin{document}


\centerline{{\large \bf Нестандартная геометрия}}
\medskip


\newcounter{prbs}
\newcommand*{\q}{\addtocounter{prbs}{1}\par\noindent\phantom{1}{\bf\arabic{prbs}.} }

\q На отрезке $AB$ отмечено 200 точек так, что весь набор симметричен относительно середины отрезка. Сто точек покрашено в синий, а остальные~--- в красный цвет. Докажите, что сумма расстояний от $A$ до красных точек равна сумме расстояний от $B$ до синих точек.

\q На квадратном столе со стороной 1 лежат 100 лоскутов, площадь каждого из которых больше $\frac{99}{100}$. Докажите, что на столе существует точка, покрытая всеми лоскутами.

\q Квадрат $2\times2$ разрезан на несколько прямоугольников. Докажите, что мы можем заштриховать несколько из них так, чтобы проекция заштрихованной фигуры на одну сторону квадрата имела длину не меньше 1, а на другую~--- не больше~1.

\q � ека Кама, протекающая в прекрасном городе Пермь, в районе порта имеет несколько островов, общий периметр которых равен~8 километрам. Один умник утверждает, что можно отчалить на лодке от порта и переправиться на другой берег, проплыв менее~3 километров. Берега реки в районе пристани параллельны, а ширина ее равна~1 километру. Прав ли умник?

\q Шесть кругов имеют общую точку. Докажите, что хотя бы один из них содержит центр некоторого другого.

\q $n$~--- нечетное число. Вершины выпуклого $n$-угольника раскрашены в несколько цветов так, что каждые две соседние вершины~--- разного цвета. Докажите, что этот $n$-угольник можно разбить на треугольники непересекающимися диагоналями, ни у одной из которых концы не окрашены одинаково.

\q В пространстве дана 8-звенная замкнутая несамопересекающаяся ломаная, вершины которой совпадают с вершинами некоторого куба. Докажите, что одно из звеньев ломаной совпадает с ребром куба.


\end{document}
