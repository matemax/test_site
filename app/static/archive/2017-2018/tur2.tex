\nopagenumbers 

\catcode`\@=11
%%
%% М кроком нд  \lines #1  для рисов ния клетч тых фигурок
%%     П р метр #1 - ширин  фигуры в клеточк х
%%
%% З клин ние вроде \cellsize=1.9ex з д ет р змер клеточек
%%
\newdimen\cellsize
\newdimen\lwidth
\cellsize=1.5ex
\bgroup

{\catcode`\|=\active \catcode`\_=\active \catcode`\ =\active% 
\catcode`\&=\active\catcode`\^^M=\active%
\gdef\linedefs{\let|\vertline\let_\horline\let \nolines\let&\phantomvertline
\let^^M\nextrow}}

\gdef\lines#1{\count@\z@\dimen0=\cellsize\advance\dimen0 .4pt
  \def\vertline{\count@\@ne\leavevmode\rlap{\vrule height \cellsize}}
  \def\phantomvertline{\count@\@ne\leavevmode\rlap{\phantom{\vrule height \cellsize}}}
  \def\horline{\count@\z@\leavevmode
    \hbox to\cellsize{\vrule height.4pt width\dimen0\hss}}
  \def\nolines{\relax\ifodd\count@\leavevmode\kern\cellsize\count@\z@
    \else\count@\@ne\fi\relax}
  \def\nextrow{\par\count@\z@}
  \lwidth\cellsize
  \multiply\lwidth by #1
  \catcode`\|=\active \catcode`\_=\active \catcode`\&=\active
  \catcode`\^^M=\active \catcode`\ =\active
  \hsize\lwidth\advance\hsize.4pt\parindent\z@\offinterlineskip\linedefs}

\egroup

\magnification=\magstep1

\hoffset=-1.5truecm

\hsize=18.5truecm

\voffset=-2truecm

\vsize=26.5truecm

\input amssym.def

\input amssym.tex

\def\geq{\geqslant}

\def\leq{\leqslant}
\def\Q{\Bbb Q}

\centerline{\bf Шестой Южный м тем тический турнир} 

\smallskip

\centerline{ВДЦ ``Орлёнок'', 12-18.09.2011 }

\smallskip

\centerline{\it Лиг  ``Гр нд'', 2 тур. 15.09.2011 }

\medskip

1. Н йдите все тройки простых чисел $p, q, r$ т кие, что
$p(p-7)+q(q-7)=r(r-7)$. 

2. Н  доске $2011 \times 2011$ стоит один кор бль в виде L-тетр мино (р спол г ться
по клетк м он может любым способом). К кое н именьшее количество выстрелов
нужно сдел ть по клетк м доски, чтобы з ведомо поп сть в кор бль?
% Чехия, 2009, второй эт п

3. Н  дуге $CD$ опис нной окружности прямоугольник  $ABCD$ взят  точк  $P$.
Точки $K$, $L$ и $M$ -- основ ния перпендикуляров, опущенных из $P$ н  прямые
$AB$, $AC$ и $BD$ соответственно. Док жите, что $\angle LKM=45^\circ$
тогд  и только тогд , когд  $ABCD$ -- кв др т.
% Т.Юрик, Чехия, 2009, фин л

4. Р внобедренный треугольник $ABC$ ($AC=BC$) впис н в окружность $k$.
Точк  $M$ лежит н  стороне $BC$. Н  луче $AM$ выбр н  точк  $N$ т к что $AN = AC$.
Окружность, опис нн я около треугольник  $MCN$, пересек ет $k$ в точк х $P$ и $C$, где
$P$ лежит н  дуге $BC$, не содерж щей точки $A$. Прямые $AB$ и $CP$ пересек ются в точке $Q$.
Док жите, что прям я $QM$ делит попол м угол между прямыми
$MB$ и $MN$. %$\angle QMB = \angle QMN $.
% Ив нов -- Болг рия

5, Вещественные числ  $a_1$, $a_2$, \dots, $a_n$ т ковы, что
$a_1+a_2+\dots+a_n=0$ и $|a_1|+|a_2|+\dots++|a_n|=~1$. Док жите, что
$|a_1+2a_2+\dots+na_n|\leq {n-1\over 2}$.
% Румыния, отбор для юниоров, 2010

6. Н йдите все п ры функций $f$, $g: \Q\to \Q$ т кие, что для всех $x, y\in \Q$
выполнены р венств 
$f(g(x)-g(y))=f(g(x))-y$ и $g(f(x)-f(y))=g(f(x))-y$.
% Греция

7. Д н пр вильный  $2n$-угольник ($n$ -- н тур льное).
В нем покр шено $2n$ ди гон лей т к, что из к ждой вершины выходит ровно
2 покр шенные ди гон ли.
Док жите, что н йдутся две п р ллельные покр шенные ди гон ли.
% Швейц рия, по мотив м з д чи В.Произволов , з д чник "Кв нт " 1993 \No 1

8. Н  плоскости н рисов н  систем  координ т (две перпендикулярные оси).
Полин  отметил  н ч ло координ т $O$,    т кже точки
с целыми координ т ми $A$ и $B$ т кие
что $O$, $A$ и $B$ являются вершин ми треугольник , внутри и н  гр нице которого
нет точек с целыми координ т ми з  исключением его вершин. После этого К тя стерл  оси координ т.
Помогите Полине по точк м $O$, $A$, $B$ восст новить оси координ т
(с помощью линейки и циркуля).
% П.Кожевников 

9. Д ны $k$ поп рно р зличных н тур льных чисел. Известно, что
НОК любых двух из них не превосходит
$10000$.
Док жите, что $k\leq 200$.

10. Решите в целых числ х ур внение $x^3y^2(2y-x)=x^2y^4-36$.
% Греция 

\vfill\eject

\centerline{\bf Шестой Южный м тем тический турнир}

\smallskip

\centerline{\bf ВДЦ "Орлёнок", 12-18.09.2011}

\smallskip

\centerline{\it Второй тур. Премьер-лиг . 15 сентября 2011 г.}

\medskip

1. Д ны вещественные числ  $a, b, c, d>0$ и $e, f, g, h<0$. Док жите, что все 
нер венств  $ae+bc>0$, $ef+cg>0$, $fd+gh>0$, $da+hb>0$ не могут выполняться 
одновременно. 
% Румыния, 2009 

2. Н  доске $12\times 12$ для игры в "морской бой" стоит один кор бль 
в виде L-тетр мино (р спол г ться по клетк м он может любым способом). 
К кое н именьшее количество выстрелов нужно сдел ть по клетк м доски, 
чтобы з ведомо поп сть в кор бль? 
% Чехия, 2009, второй эт п 

3. Н  дуге $CD$ опис нной окружности кв др т  $ABCD$ взят  точк  $P$. 
Точки $K$, $L$ и $M$ -- основ ния перпендикуляров, опущенных из $P$ н  прямые 
$AB$, $BC$ и $AD$ соответственно. Н йдите $\angle LKM$. 
% С.Берлов по мотив м з д чи чешской олимпи ды 2009 г. 

4. В остроугольном нер внобедренном треугольнике $ABC$ биссектрис  угл  $C$ 
пересек ет серединный перпендикуляр к $AB$ в точке $K$. Высоты треугольник  
$ABC$, проведенные из вершин $A$ и $B$, пересек ют отрезок $CK$ в точк х 
$P$ и $Q$ соответственно. Известно, что треугольники $AKP$ и $BKQ$ р вновелики. 
Н йдите угол $C$. 
% Чехия, 2009, второй эт п 

5. Вещественные числ  $a_1$, $a_2$, \dots, $a_n$ т ковы, что 
$a_1+a_2+\dots+a_n=0$ и $|a_1|+|a_2|+\dots++|a_n|=~1$. Док жите, что 
$|a_1+2a_2+\dots+na_n|\leq {n-1\over 2}$.
% Румыния, отбор для юниоров, 2010 

6. Н йдите все н тур льные $n$, для которых $8^n+n$ делится н  $2^n+n$. 
% Япония, 2009

7. Сумм  четырех н тур льных чисел $a$, $b$, $c$, $d$ -- простое число $p$. 
Док жите, что $ab-cd$ не делится н  $p$. 
% Румыния, 2010, 8 кл сс 

8. Т йный совет королевств  Субордин ция строго упорядочил все 10 городов королевств  
по степени их {\it зн чимости}. Госуд рственн я  ви комп ния "Зн чимые  ви линии" 
хочет соединить некоторые город  двусторонними  ви рейс ми т ким обр зом, что если город 
$A$ соединен рейсом с более зн чимым городом $B$ и $C$ зн чимее $B$, то 
между $A$ и $C$ т кже имеется рейс. Сколькими способ ми это можно сдел ть 
(в ри нт, в котором с молеты вообще не лет ют, тоже р ссм трив ется)?
% ЮАР, 2011, зн чительное упрощение  

\vfill\eject

\centerline{\bf Шестой Южный м тем тический турнир}

\smallskip

\centerline{\bf ВДЦ "Орлёнок", 12-18.09.2011}

\smallskip

\centerline{\it Второй тур. Ст рт-лиг . 15 сентября 2011 г.}

\medskip

1. В некоторой куче н стоящих монет больше, чем ф льшивых.
Все н стоящие монеты весят один ково. Люб я ф льшив я монет 
отлич ется по весу от н стоящей (вес  ф льшивых монет могут быть
р зличны). Можно ср внить вес любых двух монет при помощи ч шечных
весов, вл делец которых после к ждого взвешив ния з бир ет себе в
к честве  рендной пл ты любую (выбр нную им) монету из только что
взвешенных. Верно ли, что можно выделить хотя бы одну н стоящую
монету и ост вить ее себе?
% А.Ш пов лов 

2. Д н остроугольный р внобедренный треугольник $ABC$ с вершиной $C$.
Пусть $A'$ -- основ ние высоты, опущенной из вершины $A$. Ок з лось,
что $CA'={1\over 2}AB$. Док жите, что треугольник $ABC$
р вносторонний.
% предложил Б.Трушин 

3. Д но восемь кр сных и по четыре синих и зеленых ш р . Сколько
существует способов выложить их в ряд т к, чтобы одноцветные ш ры не
стояли рядом?

4. Д ны вещественные числ  $a, b, c, d > 0$ и $e, f, g, h < 0$. Док жите,
что все нер венств  $ae + bc > 0$, $ef + cg > 0$, $fd + gh > 0$, $da
+ hb > 0$ не могут выполняться одновременно.
% Румыния, 2009 

5. Известно, что сумм  кубов трех н тур льных чисел может быть кубом
н тур льного числ  (н пример, $1^3 + 6^3 + 8^3 = 9^3$). А существуют
ли 2011 поп рно р зличных н тур льных чисел т ких, что сумм  их
кубов р вн  кубу некоторого н тур льного числ ?
% Жюри 

6. Н  доске $12\times 12$ стоит один кор бль в виде 
\leavevmode \vbox{\lines3
 _  
|_|_ _
|_|_|_|}
(р спол г ться по клетк м он может любым способом). К кое н именьшее
количество выстрелов нужно сдел ть по клетк м доски, чтобы з ведомо
поп сть в кор бль?
% Чехия, 2009, второй эт п 

7. Сумм  четырех н тур льных чисел $a$,
$b$, $c$, $d$ -- простое число $p$. Док жите, что $ab-cd$ не делится
н  $p$.
% Румыния, 2010, 8 кл сс 

8. Числ  $x_k$ при н тур льных $k$ з д ны р венством $x_k = 2^k -
k$. Н йдите все н тур льные $n$, для которых сумм  $1 + x_1 + x_2 +
\ldots + x_n$ является степенью двойки с н тур льным пок з телем.
% Румыния, 2010, уездный тур, 9 кл сс 

\end 
